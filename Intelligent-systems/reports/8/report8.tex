\documentclass[10pt,a4paper]{ltjsarticle}       % LuaTeX を使う
\usepackage[luatex]{graphicx}             % LuaTeX 用, draft がついているときは図の代わりに同じ大きさの枠ができる
\usepackage{here}                               % 図表の位置を強制して出力
\usepackage{afterpage}                          % 残っている図を貼り付ける(\afterpage{\clearpage})
\usepackage[subrefformat=parens]{subcaption}    % サブキャプション(図1(a) とか)
\usepackage{setspace}                           % 行間制御
\usepackage{ulem}                               % 下線や取り消し線など
\usepackage{booktabs}                           % きれいな表(\toprule \midrule \bottomrule)
\usepackage{multirow}                           % 表で行結合
\usepackage{multicol}                           % 表で列結合
\usepackage{hhline}                             % 表で 2 重線
\usepackage[table]{xcolor}                      % カラー
\usepackage{tikz}                               % 図描画用
\usepackage[framemethod=tikz]{mdframed}         % 文章を囲むとき用
\usepackage[version=3]{mhchem}                  % 化学式
\usepackage{siunitx}                            % 単位
\usepackage{comment}                            % コメント
\setcounter{tocdepth}{3}                        % 目次に subsubsection まで表示
\usepackage{listings}
\lstdefinelanguage{Julia}%
  {morekeywords={abstract,break,case,catch,const,continue,do,else,elseif,%
      end,export,false,for,function,immutable,import,importall,if,in,%
      macro,module,otherwise,quote,return,switch,true,try,type,typealias,%
      using,while},%
   sensitive=true,%
   alsoother={\$},%
   morecomment=[l]\#,%
   morecomment=[n]{\#=}{=\#},%
   morestring=[s]{"}{"},%
   morestring=[m]{'}{'},%
}[keywords,comments,strings]%

\lstset{%
    language         = Julia,
    basicstyle       = \ttfamily,
    keywordstyle     = \bfseries\color{blue},
    stringstyle      = \color{magenta},
    commentstyle     = \color{ForestGreen},
    showstringspaces = false,
}
%\lstset{
%    frame=single,
%    basicstyle=\small\ttfamily,
%    tabsize=4,
%    language=python,
%    keywordstyle=\color{red},
%    stringstyle=\color{blue}
%}

% -----ヘッダ・フッタの設定-----
\usepackage{fancyhdr}
\usepackage{lastpage}
\pagestyle{fancy}
\lhead{}                                 % 左ヘッダ
\chead{}                                 % 中央ヘッダ
\rhead{}                                 % 右ヘッダ
\lfoot{}                                 % 左フッタ
\cfoot{\thepage~/~\pageref{LastPage}}    % 中央フッタ
\rfoot{}                                 % 右フッタ
\renewcommand{\headrulewidth}{0pt}       % ヘッダの罫線を消す
% -----余白の設定-----
% これをアンコメントするとページ番号が中央からずれるから今は使わない.
% \usepackage[left=19.05mm,right=19.05mm,top=25.40mm,bottom=25.40mm]{geometry}
% -----フォントの設定-----
% https://ja.osdn.net/projects/luatex-ja/wiki/LuaTeX-ja%E3%81%AE%E4%BD%BF%E3%81%84%E6%96%B9
% http://myfuturesightforpast.blogspot.jp/2013/12/tex-gyre.html など
\usepackage[no-math]{fontspec}
\usepackage{amsmath,amssymb}    % 高度な数式用
\usepackage{mathrsfs}           % 花文字用
% times ベース -> txfonts
% palatino ベース -> pxfonts
\usepackage{txfonts}
\usepackage{bm}                 % 斜体太字ベクトル
% Avant Garde -> TeX Gyre Adventor
% Bookman Old Style -> TeX Gyre Bonum
% Zapf Chancery -> TeX Gyre Chorus
% Courier -> TeX Gyre Cursor
% Helvetica -> TeX Gyre Heros
% Helvetica Narrow -> TeX Gyre Heros Cn
% Palatino -> TeX Gyre Pagella
% New Century Schoolbook -> TeX Gyre Schola
% Times -> TeX Gyre Termes
\setmainfont[Ligatures=TeX]{TeXGyreTermes}
\setsansfont[Ligatures=TeX]{TeXGyreHeros}
\setmonofont[Scale=MatchLowercase]{TeXGyreCursor}
\usepackage[match,deluxe,expert,bold]{luatexja-fontspec}
\setmainjfont[BoldFont=IPAexGothic]{IPAexMincho}
\setsansjfont{IPAexGothic}
\usepackage{luatexja-otf}
% -----PDF ハイパーリンク,ブックマーク,URL の設定-----
% オプション(\hypersetup{})は https://texwiki.texjp.org/?hyperref 参照
\usepackage{url}
% -----ソースコードの設定-----
% オプション(\lstset{})は http://tug.ctan.org/tex-archive/macros/latex/contrib/listings/listings.pdf 参照
% 使うときは
% \begin{lstlisting}[language=aaaa,caption=bbbb,label=List:cccc]
% hogehoge
% \end{lstlisting}
\usepackage{listings}
\lstset{%
  basicstyle=\ttfamily\small,%
  frame=single,%
  frameround=ffff,%
  numbers=left,%
  stepnumber=1,%
  numbersep=1\zw,%
  breaklines=true,%
  tabsize=4,%
  captionpos=t,%
  commentstyle=\itshape}
% -----図表等の reference の設定-----
% 表示文字列を日本語化
\renewcommand{\figurename}{図}
\renewcommand{\tablename}{表}
\renewcommand{\lstlistingname}{リスト}
\renewcommand{\abstractname}{概要}
% 図番号等を"<章番号>.<図番号>"
% lstlisting に関しては https://tex.stackexchange.com/questions/134418/numbering-of-listings 参照
\renewcommand{\thefigure}{\thesection.\arabic{figure}}
\renewcommand{\thetable}{\thesection.\arabic{table}}
\AtBeginDocument{\renewcommand{\thelstlisting}{\thesection.\arabic{lstlisting}}}
\renewcommand{\theequation}{\thesection.\arabic{equation}}
% 節が進むごとに図番号等をリセット
% http://d.hatena.ne.jp/gp98/20090919/1253367749 参照
\makeatletter
\@addtoreset{figure}{section}
\@addtoreset{table}{section}
\@addtoreset{lstlisting}{section}
\@addtoreset{equation}{section}
\makeatother
% \ref{} の簡単化
\newcommand*{\refSec}[1]{\ref{#1}~章}
\newcommand*{\refSsec}[1]{\ref{#1}~節}
\newcommand*{\refSssec}[1]{\ref{#1}~項}
\newcommand*{\refFig}[1]{\figurename~\ref{#1}}
\newcommand*{\refTab}[1]{\tablename~\ref{#1}}
\newcommand*{\refList}[1]{\lstlistingname~\ref{#1}}
\newcommand*{\refEq}[1]{式~(\ref{#1})}
% -----数式中便利な定義-----
% https://www.library.osaka-u.ac.jp/doc/TA_LaTeX2.pdf
% https://en.wikibooks.org/wiki/LaTeX/Mathematics など
\newcommand{\e}{\mathrm{e}}                     % ネイピア数
\newcommand{\imagi}{\mathrm{i}}                 % 虚数単位(i)
\newcommand{\imagj}{\mathrm{j}}                 % 虚数単位(j)
\newcommand{\vDel}{\varDelta}                   % デルタ大文字
\newcommand{\veps}{\varepsilon}                 % イプシロン小文字
\newcommand*{\paren}[1]{\left( #1 \right)}      % () を中身の大きさに合わせる
\newcommand*{\curly}[1]{\left\{ #1 \right\}}    % {} を中身の大きさに合わせる
\newcommand*{\bracket}[1]{\left[ #1 \right]}    % [] を中身の大きさに合わせる
\renewcommand{\Re}{\operatorname{Re}}           % 実部
\renewcommand{\Im}{\operatorname{Im}}           % 虚部
\newcommand*\sfrac[2]{{}^{#1}\!/_{#2}}          % xfrac パッケージの \sfrac{}{} の代わり
\renewcommand*\vec[1]{\mathbf{#1}}              % 矢印ベクトルは使わないので上書き.太字立体.
\newcommand{\argmax}{\mathop{\rm arg~max}\limits}
\newcommand{\argmin}{\mathop{\rm arg~min}\limits}

\title{知的システム論第8回レポート}
\author{37186305\\航空宇宙工学専攻修士一年\\荒居秀尚}

\begin{document}
\maketitle
\section{宿題1}
Julia 1.0.0で貪欲探索と最適探索を実装した。\\
実装にあたって、迷路はノード間の長さを重みとして与えた隣接行列の形で表現した。
\subsection{貪欲探索}
\begin{lstlisting}
mutable struct edge
    from::Int64
    to::Int64
    cost::Int64
end

function greedy(m)
    openlist = [edge(1, 1, 0)]
    closedlist = []
    totalcost = 0
    while length(openlist) > 0
        s = pop!(openlist)
        push!(closedlist, s)
        totalcost += s.cost
        if size(m)[1] == s.to
            break
        end
        candidate = [i for i in 1:length(m[s.to, :]) if m[s.to, i] != 0]
        next = reverse(sort(candidate, by=x -> m[s.to, x]))
        closed_from = [n.from for n in closedlist]
        for n in next
            if n in closed_from
                continue
            end
            push!(openlist, edge(s.to, n, m[s.to, n]))
        end
    end
    closedlist, totalcost
end

m = [0 1 0 2 0 0 0 0 0 0;
     1 0 2 0 0 0 0 0 0 0;
     0 2 0 0 0 0 1 0 0 0;
     2 0 0 0 1 0 0 0 0 0;
     0 0 0 1 0 2 0 0 0 0;
     0 0 0 0 2 0 1 0 0 0;
     0 0 1 0 0 1 0 1 0 0;
     0 0 0 0 0 0 1 0 1 2;
     0 0 0 0 0 0 0 1 0 0;
     0 0 0 0 0 0 0 2 0 0]

greedy(m)
\end{lstlisting}
 これを実行すると、通る経路としては
\begin{equation*}
A \rightarrow B \rightarrow C \rightarrow G \rightarrow F \rightarrow E \rightarrow D \rightarrow H \rightarrow I \rightarrow J
\end{equation*}
となる。ただし分岐に関しては、遡りは行わないこととした。この時ゴールまでのコストは12であった。
\subsection{最適探索}
\begin{lstlisting}
import Base: sort


mutable struct node
    id::Int64
    min_cost::Int64
    from::Int64
end

function sort(a::Array{node, 1})
    costs = [n.min_cost for n in a]
    idx = sortperm(costs)
    a = a[idx]
end

function dijkstra(m)
    initnode = node(1, 0, 1)
    pendinglist = [initnode]
    allnodes = []
    closedlist = []
    while true
        pending_id = [n.id for n in pendinglist]
        if size(m)[1] in pending_id
            break
        end

        pendinglist = sort(pendinglist)
        s = popfirst!(pendinglist)
        all_id = [n.id for n in allnodes]
        if !(s.id in all_id)
            push!(allnodes, s)
        end
        idx = s.id
        root = m[idx, :]
        candidate_idx = [i for i in 1:length(root) if root[i] != 0]
        pending_id = [n.id for n in pendinglist]
        for i in candidate_idx
            cost = s.min_cost + root[i]
            if i in pending_id
                pending_idx = indexin(i, pending_id)[1]
                if cost < (pendinglist[pending_idx].min_cost)
                    pendinglist[pending_idx].min_cost = cost
                    pendinglist[pending_idx].from = idx
                end
                continue
            end
            push!(pendinglist, node(i, cost, idx))
        end
    end
    
    pending_id = [n.id for n in pendinglist]
    all_id = [n.id for n in allnodes]
    poslast = indexin(size(m)[1], pending_id)[1]
    current = pendinglist[poslast]
    push!(closedlist, current)
    while true
        if current.from == 1
            push!(closedlist, initnode)
            break
        end
        
        pos = indexin(current.from, all_id)[1]
        push!(closedlist, allnodes[pos])
        current = allnodes[pos]
    end
    sort(closedlist)
end

m = [0 1 0 2 0 0 0 0 0 0;
     1 0 2 0 0 0 0 0 0 0;
     0 2 0 0 0 0 1 0 0 0;
     2 0 0 0 1 0 0 0 0 0;
     0 0 0 1 0 2 0 0 0 0;
     0 0 0 0 2 0 1 0 0 0;
     0 0 1 0 0 1 0 1 0 0;
     0 0 0 0 0 0 1 0 1 2;
     0 0 0 0 0 0 0 1 0 0;
     0 0 0 0 0 0 0 2 0 0]

dijkstra(m)
\end{lstlisting}
これを実行すると、最終的な経路としては
\begin{equation*}
A \rightarrow B \rightarrow C \rightarrow G \rightarrow H \rightarrow J
\end{equation*}
のコスト7の経路を発見することができることが確かめられた。また、このdijkstra(m)内部で最終的なallnodes配列を出力してみると、
\begin{equation*}
A \rightarrow D \rightarrow E \rightarrow F
\end{equation*}
の経路も$G \rightarrow H$を計算する前に検証していることがわかる。
\section{宿題2}
Julia 1.0.0で実装した。
\begin{lstlisting}
import Base: sort, indexin


function options(p)
    idx = indexin(0, p)[1]
    indices = [CartesianIndex(i, j) for (i, j) in [
                (idx[1]-1, idx[2]), (idx[1]+1, idx[2]),
                (idx[1], idx[2]-1), (idx[1], idx[2]+1)
                ]]
    indices = [i for i in indices if 0 < i[1] <= size(p)[1] && 0 < i[2] <= size(p)[2]]
end


function flip(p, idx)
    zeropos = indexin(0, p)[1]
    p[zeropos], p[idx] = p[idx], p[zeropos]
    p
end


function number_of_misplace(p, o)
    sum(sign.(abs.(p - o)))
end


function manhattan(p, o)
    distance = 0
    for i in 1:maximum(p)
        idxp = indexin(i, p)[1]
        idxo = indexin(i, o)[1]
        cartesian_diff = idxp - idxo
        distance += abs(cartesian_diff[1]) + abs(cartesian_diff[2])
    end
    distance
end


mutable struct State
    state::Array{Int64, 2}
    before::Array{Int64, 2}
    min_cost::Int64
    heuristic::Int64
end


function sort(a::Array{State, 1})
    costs = [s.min_cost + s.heuristic for s in a]
    a[sortperm(costs)]
end


function indexin(elem::Array{Int64, 2}, arr::Array{Array{Int64, 2}, 1})
    [i for i in 1:length(arr) if arr[i] == elem][1]
end


function astar(p, o, f)
    pendinglist = [State(copy(p), copy(p), 0, f(p, o))]
    visitedlist = []
    closedlist = []
    while true
        pendinglist = sort(pendinglist)
        next = popfirst!(pendinglist)
        visited_state = [s.state for s in visitedlist]
        if !(next.state in visited_state)
            push!(visitedlist, next)
        end
        next_option = options(next.state)
        next_p = [flip(copy(next.state), i) for i in next_option]
        next_state = [State(np, copy(next.state), next.min_cost+1, f(np, o)) for np in next_p]
        if o in next_p
            push!(visitedlist, State(o, copy(next.state), next.min_cost+1, f(o, o)))
            break
        end
        
        pending_state = [s.state for s in pendinglist]
        for ns in next_state
            if !(ns.state in pending_state)
                push!(pendinglist, ns) 
            end
        end
    end
    visited_state = [s.state for s in visitedlist]
    current = o
    while true
        idxcurrent = indexin(current, visited_state)
        push!(closedlist, visitedlist[idxcurrent])
        if current == p
            break
        end
        before = visitedlist[idxcurrent].before
        idxbefore = indexin(before, visited_state)
        current = visitedlist[idxbefore].state
    end
    reverse(closedlist)
end


puzzle = [4 3 5;
          2 1 0]

optimal = [1 2 3;
           4 5 0]

astar(puzzle, optimal, number_of_misplace)
\end{lstlisting}
これを実行すると、最終的な経路としては、
\begin{equation*}
\left[
\begin{matrix}
  4 & 3 & 5\\
  2 & 1 & 0
\end{matrix}
\right] \rightarrow \left[
\begin{matrix}
  4 & 3 & 0\\
  2 & 1 & 5
\end{matrix}
\right] \rightarrow \left[
\begin{matrix}
  4 & 0 & 3\\
  2 & 1 & 5
\end{matrix}
\right] \rightarrow 
\left[
\begin{matrix}
  4 & 1 & 3\\
  2 & 0 & 5    
\end{matrix}
\right] \rightarrow \left[
\begin{matrix}
  4 & 1 & 3\\
  0 & 2 & 5
\end{matrix}
\right] \rightarrow \left[
\begin{matrix}
  0 & 1 & 3\\
  4 & 2 & 5
\end{matrix}
\right] \rightarrow 
\left[
\begin{matrix}
  1 & 0 & 3\\
  4 & 2 & 5
\end{matrix}
\right] \rightarrow \left[
\begin{matrix}
  1 & 2 & 3\\
  4 & 0 & 5
\end{matrix}
\right] \rightarrow \left[
\begin{matrix}
  1 & 2 & 3\\
  4 & 5 & 0
\end{matrix}
\right]
\end{equation*}
が出力される。なお、上のプログラム例では誤りタイルの数をヒューリスティック評価値としているが、これをマンハッタン距離に変えても全く同じ結果が得られる。
\end{document}