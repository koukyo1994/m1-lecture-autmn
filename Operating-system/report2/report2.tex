\documentclass[10pt,a4paper]{ltjsarticle}       % LuaTeX を使う
\usepackage[luatex]{graphicx}             % LuaTeX 用, draft がついているときは図の代わりに同じ大きさの枠ができる
\usepackage{here}                               % 図表の位置を強制して出力
\usepackage{afterpage}                          % 残っている図を貼り付ける(\afterpage{\clearpage})
\usepackage[subrefformat=parens]{subcaption}    % サブキャプション(図1(a) とか)
\usepackage{setspace}                           % 行間制御
\usepackage{ulem}                               % 下線や取り消し線など
\usepackage{booktabs}                           % きれいな表(\toprule \midrule \bottomrule)
\usepackage{multirow}                           % 表で行結合
\usepackage{multicol}                           % 表で列結合
\usepackage{hhline}                             % 表で 2 重線
\usepackage[table]{xcolor}                      % カラー
\usepackage{tikz}                               % 図描画用
\usepackage[framemethod=tikz]{mdframed}         % 文章を囲むとき用
\usepackage[version=3]{mhchem}                  % 化学式
\usepackage{siunitx}                            % 単位
\usepackage{comment}                            % コメント
\setcounter{tocdepth}{3}                        % 目次に subsubsection まで表示
\usepackage{listings}
\lstset{
    frame=single,
    basicstyle=\small\ttfamily,
    tabsize=4,
    language=python,
    keywordstyle=\color{red},
    stringstyle=\color{blue}
}

% -----ヘッダ・フッタの設定-----
\usepackage{fancyhdr}
\usepackage{lastpage}
\pagestyle{fancy}
\lhead{}                                 % 左ヘッダ
\chead{}                                 % 中央ヘッダ
\rhead{}                                 % 右ヘッダ
\lfoot{}                                 % 左フッタ
\cfoot{\thepage~/~\pageref{LastPage}}    % 中央フッタ
\rfoot{}                                 % 右フッタ
\renewcommand{\headrulewidth}{0pt}       % ヘッダの罫線を消す
% -----余白の設定-----
% これをアンコメントするとページ番号が中央からずれるから今は使わない.
% \usepackage[left=19.05mm,right=19.05mm,top=25.40mm,bottom=25.40mm]{geometry}
% -----フォントの設定-----
% https://ja.osdn.net/projects/luatex-ja/wiki/LuaTeX-ja%E3%81%AE%E4%BD%BF%E3%81%84%E6%96%B9
% http://myfuturesightforpast.blogspot.jp/2013/12/tex-gyre.html など
\usepackage[no-math]{fontspec}
\usepackage{amsmath,amssymb}    % 高度な数式用
\usepackage{mathrsfs}           % 花文字用
% times ベース -> txfonts
% palatino ベース -> pxfonts
\usepackage{txfonts}
\usepackage{bm}                 % 斜体太字ベクトル
% Avant Garde -> TeX Gyre Adventor
% Bookman Old Style -> TeX Gyre Bonum
% Zapf Chancery -> TeX Gyre Chorus
% Courier -> TeX Gyre Cursor
% Helvetica -> TeX Gyre Heros
% Helvetica Narrow -> TeX Gyre Heros Cn
% Palatino -> TeX Gyre Pagella
% New Century Schoolbook -> TeX Gyre Schola
% Times -> TeX Gyre Termes
\setmainfont[Ligatures=TeX]{TeXGyreTermes}
\setsansfont[Ligatures=TeX]{TeXGyreHeros}
\setmonofont[Scale=MatchLowercase]{TeXGyreCursor}
\usepackage[match,deluxe,expert,bold]{luatexja-fontspec}
\setmainjfont[BoldFont=IPAexGothic]{IPAexMincho}
\setsansjfont{IPAexGothic}
\usepackage{luatexja-otf}
% -----PDF ハイパーリンク,ブックマーク,URL の設定-----
% オプション(\hypersetup{})は https://texwiki.texjp.org/?hyperref 参照
\usepackage{url}
% -----ソースコードの設定-----
% オプション(\lstset{})は http://tug.ctan.org/tex-archive/macros/latex/contrib/listings/listings.pdf 参照
% 使うときは
% \begin{lstlisting}[language=aaaa,caption=bbbb,label=List:cccc]
% hogehoge
% \end{lstlisting}
\usepackage{listings}
\lstset{%
    basicstyle=\ttfamily\small,%
    frame=single,%
    frameround=ffff,%
    numbers=left,%
    stepnumber=1,%
    numbersep=1\zw,%
    breaklines=true,%
    tabsize=4,%
    captionpos=t,%
    commentstyle=\itshape}
% -----図表等の reference の設定-----
% 表示文字列を日本語化
\renewcommand{\figurename}{図}
\renewcommand{\tablename}{表}
\renewcommand{\lstlistingname}{リスト}
\renewcommand{\abstractname}{概要}
% 図番号等を"<章番号>.<図番号>"
% lstlisting に関しては https://tex.stackexchange.com/questions/134418/numbering-of-listings 参照
\renewcommand{\thefigure}{\thesection.\arabic{figure}}
\renewcommand{\thetable}{\thesection.\arabic{table}}
\AtBeginDocument{\renewcommand{\thelstlisting}{\thesection.\arabic{lstlisting}}}
\renewcommand{\theequation}{\thesection.\arabic{equation}}
% 節が進むごとに図番号等をリセット
% http://d.hatena.ne.jp/gp98/20090919/1253367749 参照
\makeatletter
\@addtoreset{figure}{section}
\@addtoreset{table}{section}
\@addtoreset{lstlisting}{section}
\@addtoreset{equation}{section}
\makeatother
% \ref{} の簡単化
\newcommand*{\refSec}[1]{\ref{#1}~章}
\newcommand*{\refSsec}[1]{\ref{#1}~節}
\newcommand*{\refSssec}[1]{\ref{#1}~項}
\newcommand*{\refFig}[1]{\figurename~\ref{#1}}
\newcommand*{\refTab}[1]{\tablename~\ref{#1}}
\newcommand*{\refList}[1]{\lstlistingname~\ref{#1}}
\newcommand*{\refEq}[1]{式~(\ref{#1})}
% -----数式中便利な定義-----
% https://www.library.osaka-u.ac.jp/doc/TA_LaTeX2.pdf
% https://en.wikibooks.org/wiki/LaTeX/Mathematics など
\newcommand{\e}{\mathrm{e}}                     % ネイピア数
\newcommand{\imagi}{\mathrm{i}}                 % 虚数単位(i)
\newcommand{\imagj}{\mathrm{j}}                 % 虚数単位(j)
\newcommand{\vDel}{\varDelta}                   % デルタ大文字
\newcommand{\veps}{\varepsilon}                 % イプシロン小文字
\newcommand*{\paren}[1]{\left( #1 \right)}      % () を中身の大きさに合わせる
\newcommand*{\curly}[1]{\left\{ #1 \right\}}    % {} を中身の大きさに合わせる
\newcommand*{\bracket}[1]{\left[ #1 \right]}    % [] を中身の大きさに合わせる
\renewcommand{\Re}{\operatorname{Re}}           % 実部
\renewcommand{\Im}{\operatorname{Im}}           % 虚部
\newcommand*\sfrac[2]{{}^{#1}\!/_{#2}}          % xfrac パッケージの \sfrac{}{} の代わり
\renewcommand*\vec[1]{\mathbf{#1}}              % 矢印ベクトルは使わないので上書き.太字立体.
\newcommand{\argmax}{\mathop{\rm arg~max}\limits}
\newcommand{\argmin}{\mathop{\rm arg~min}\limits}

\title{オペレーティングシステム特論第四回課題}
\author{37186305\\航空宇宙工学専攻修士一年\\荒居秀尚}

\begin{document}
  \maketitle
  \section{課題1}
  子プロセスのVMA情報を取得して表示した。
  \begin{lstlisting}
  #include <stdio.h>
  #include <stdlib.h>
  #include <unistd.h>
  #include <sys/types.h>
  #include <sys/wait.h>

  int main(void) {
      pid_t pid = -42;
      int wstatus = -42;
      int ret = -1;
      int c;
      FILE *fd;
      char str[128];

      pid = fork();
      switch (pid) {
          case -1:
              perror("fork");
              return EXIT_FAILURE;
          case 0:
              sleep(1);
              printf("Noooooo!\n");
              exit(0);
          default:
              printf("Iamyourfather!\n");
              sprintf(str, "/proc/%d/maps", pid);
              fd = fopen(str, "r");
              if (!fd) {
                  perror("fopen");
                  break;
              }
              while ((c = fgetc(fd)) != EOF) {
                  if (putchar(c) < 0) {
                      fclose(fd);
                      break;
                  }
              }
              fclose(fd);
              break;
      }

      ret = waitpid(pid, &wstatus, 0);
      if (ret == -1) {
          perror("waitpid");
          return EXIT_FAILURE;
      }
      printf("Childexitstatus: %d\n",
             WEXITSTATUS(wstatus));
      return EXIT_SUCCESS;
  }
  \end{lstlisting}
  結果は、以下のようになった。
  \begin{lstlisting}
  Noooooo!
  Iamyourfather!
  00400000-00401000 r-xp 00000000 08:08 17695897                           /home/hidehisa/master/m1-lecture-autmn/Operating-system/report2/process
  00600000-00601000 r--p 00000000 08:08 17695897                           /home/hidehisa/master/m1-lecture-autmn/Operating-system/report2/process
  00601000-00602000 rw-p 00001000 08:08 17695897                           /home/hidehisa/master/m1-lecture-autmn/Operating-system/report2/process
  7ff65a931000-7ff65aaf1000 r-xp 00000000 08:08 25952340                   /lib/x86_64-linux-gnu/libc-2.23.so
  7ff65aaf1000-7ff65acf1000 ---p 001c0000 08:08 25952340                   /lib/x86_64-linux-gnu/libc-2.23.so
  7ff65acf1000-7ff65acf5000 r--p 001c0000 08:08 25952340                   /lib/x86_64-linux-gnu/libc-2.23.so
  7ff65acf5000-7ff65acf7000 rw-p 001c4000 08:08 25952340                   /lib/x86_64-linux-gnu/libc-2.23.so
  7ff65acf7000-7ff65acfb000 rw-p 00000000 00:00 0
  7ff65acfb000-7ff65ad21000 r-xp 00000000 08:08 25952281                   /lib/x86_64-linux-gnu/ld-2.23.so
  7ff65aeef000-7ff65aef2000 rw-p 00000000 00:00 0
  7ff65af20000-7ff65af21000 r--p 00025000 08:08 25952281                   /lib/x86_64-linux-gnu/ld-2.23.so
  7ff65af21000-7ff65af22000 rw-p 00026000 08:08 25952281                   /lib/x86_64-linux-gnu/ld-2.23.so
  7ff65af22000-7ff65af23000 rw-p 00000000 00:00 0
  7ffc11340000-7ffc11362000 rw-p 00000000 00:00 0                          [stack]
  7ffc11384000-7ffc11387000 r--p 00000000 00:00 0                          [vvar]
  7ffc11387000-7ffc11389000 r-xp 00000000 00:00 0                          [vdso]
  ffffffffff600000-ffffffffff601000 r-xp 00000000 00:00 0                  [vsyscall]
  Childexitstatus: 0
  \end{lstlisting}
\end{document}
